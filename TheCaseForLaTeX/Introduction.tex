% What is LaTeX?
% Formatting stuff in MS Word can waste a lot of time - LaTeX fixes this.
	% All formatting is defined in one place - only the raw content needs to be provided.
	% Example of Word being a nightmare: https://learn.microsoft.com/en-us/answers/questions/5092571/tables-in-ms-word-are-changing-when-saved-and-re-o
	% Documents are more consistent.
% LaTeX is much easier for technical documents.
	% Equations are easy.
	% Figures are input by reference - no need to reinsert figures when they are updated.
	% Vector image formats are handled easily - higher quality and smaller file sizes for plots.
	% Cross-referencing and citations are easy.
% Ease of use
	% Once the template has been defined it is simple to use.
	% Has good documentation and community on Stack Exchange.
	% Well-established - 40 years of use and widely used in academia.
% Other
	% RESEARCH THIS - Works well with version control systems like Git. Better for traceability
	% It is free and open source.
	% Handles subfigures.
	
\begin{center}
	\huge The Case for \LaTeX\  \\
	\vspace{5mm}
	\large Henry Ginn  \\
	\vspace{3 mm}
	\large \today
\end{center}

\section{Introduction}
\label{Introduction}

This document gives a brief overview of what \LaTeX\ (pronounced ``Lay-tek") is and highlights some of the advantages it has over Microsoft Word. \LaTeX\ is a well-established piece of typesetting software for producing technical documents where the input is given in plain text which is then compiled into a PDF. It is understandable why one may be wary of any potential barriers to those less comfortable with technology, but the payoff is more than worth it, and \LaTeX\ is far simpler than it may seem.

The main advantages are as follows:

\begin{itemize}
	\item Less buggy than Word
	\item No need to worry about formatting - template is followed automatically
	\item Equations, cross-references, and citations are far easier than in Word
	\item Vector image compatible
	\item Images are given by reference
	\item The \LaTeX\ compiler knows how to typeset better than you so documents look professional
\end{itemize}

In section~\ref{sec:Formatting} we elaborate on why \LaTeX\ formatting is easier than in Word, followed by the advantages when writing technical documents in section~\ref{sec:TechnicalWriting}. We finish with a discussion on the accessibility of \LaTeX\ in section~\ref{sec:EaseOfUse} before concluding, with an appendix demonstrating why Word can be frustrating to work.












