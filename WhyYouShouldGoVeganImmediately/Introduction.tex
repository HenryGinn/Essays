\section{Introduction}

This essay argues for ethical veganism. This is the position that the consumption of animals is not justified from an ethical perspective and therefore individuals should boycott the animal agriculture industry. This is not an argument against the consumption of only meat, a non-total reduction in consumption, or for higher welfare standards on farms and slaughterhouses. We argue that those without extreme extenuating circumstances\footnote{Such circumstances include those with many allergies or significant dietary requirements where totally abstaining is not reasonably viable, those on desert island with no other choice, inuit fisherpeople, and so on.} should be completely eliminating all animal products from their diet and lifestyle\footnote{Sometimes this cannot be practically avoided, for example, UK bank notes and most batteries contain animal products. For convenience we will not list these obvious exclusions when using ``all".}. Only arguments concerning ethics will be presented here, although there are many other strong reasons for veganism outside of this scope.

Engaging with this essay is not an academic exercise. It is not something to be read and then moved on from, and at the end the reader will need to make an important decision about who they are as a person and what their future lifestyle will be. This is not a topic to stick your head in the sand about.

Before we begin, we outline the position being argued against. The main reasons to be non-vegan are the taste pleasure experienced by eating products derived from animal sources, the nutritional convenience of an unrestricted diet, and the social convenience of fitting in with everyone else in. We claim that these reasons are not sufficiently strong to enough justify the treatment of animals that support a non-vegan lifestyle. We look at these reasons closer in subsection~\ref{sec:TheCostOfVeganism}, but for now the reader is invited to keep these points in mind and compare them with each presented argument.

There are often two stages in becoming an ethical vegan. The first is to be intellectually convinced where it is understood that one should go vegan but one may not be ready to make the transition. The second is being emotionally convinced where one believes they have no choice but to go vegan if they are to continue considering themselves an ethical person. These are covered in sections~\ref{sec:PhilosophicalArguments} and~\ref{sec:EmotionalArguments} respectively, followed by a discussion on some common counter-arguments in section~\ref{sec:CounterArguments}.

% There are two main attitudes to ethical veganism. The first is aiming for the minimisation of suffering where practicable. This is the position held by the vegan society, and actions such as eating honey and riding horses can be seen as justified from this position. The second is one from principle - animals are not there for our exploitation or consumption, end of story. This is much harder to defend, although the position I personally take.