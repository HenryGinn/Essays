% Introduction. This section aims to do two things: a) demonstrate that animals have moral worth, and b) this moral worth is sufficient to not treat them the way we do.
% Name the trait
    % Explain why you are not racist or homophobic - why can you not apply a similar argument to animals?
% Veil of ignorance
% Do you care about minimising suffering - this entails going vegan
% How much moral worth do animals have
    % What determines moral worth?
    % Where would you place particular animals on a spectrum from a rock to a human? How comfortable would you be hitting either with a hammer?
% Philosophical consistency. Are you against oppression?

% I'll need to define moral worth


\section{Philosophical Arguments}
\label{sec:PhilosophicalArguments}

This section aims to show that the difference in how we treat humans and non-human animals is not justified. We aim to show that the gap is far smaller than what is currently accepted, and that the moral worth of non-human animals is sufficiently high to significantly change how we treat them. This is not a philosophy~101 course and it is assumed that the reader does not need convincing of the claim that humans have moral worth and that human suffering is bad as a result. We argue that this should be extended to the claim that suffering is bad, including the suffering of non-human animals.

We will define the moral worth of an entity as how much weight that entity deserves in decisions about morality. Consider a spectrum with a rock at 0\% and humans at 100\%. We can agree that it would be morally indifferent to hit a rock with a hammer, and morally reprehensible to bludgeon a human with a hammer. We do not believe the rock has any moral interests that need to be considered. Some have argued that animals function as mindless automota like a mechanical devices, and therefore animals are morally equivalent to rocks. We nip this objection in the bud by dismissing it out of hand because this belief is patently ridiculous.

\subsection{Name the Trait}

We claim that the only morally relevant traits are the ability to experience pain and pleasure. We give some examples of other traits that appear relevant and demonstrate why these do not align with our moral intuitions. The challenge is to find a morally relevant trait that separates.

As a trivial example to show how the argument goes, suppose we were to claim that the reason human suffering was bad was because we had prehensile digits. We could consider a human with no hands and all agree that mercilessly torturing such a person would be a bad thing. From this we conclude that the moral worth of humans is not dependent on having prehensile digits.

The goal will be to identify the trait that separates humans from animals. Some claim the trait is the ability to reason, although we can find severely intellectually disabled humans with lower reasoning abilities than pigs, crows, or even a smart chicken. We are not about to claim that those with profound intellectual disabilities can be treated with wanton abandon. We still believe that bludgeoning such a person to death would be morally reprehensible, therefore we go the other way and conclude that a being does not need to have the ability to reason to have moral worth.

Other proposed traits include understanding one's self as a being that has a past, present, and future, having an internal thought process, having the ability to communicate with others, and many more. For all of these it is relatively simply to find or construct examples of humans that do not possess such a trait and non-human animals that do. We challenge the reader to find such a trait or combinations of traits, and keep in mind that its presence will need to explain the chasm that we put between humans and non-human animals in terms of moral worth.

The only trait that one may find is belonging to the species of homo sapiens. It certainly satisfies the conditions that all humans possess it and that all non-humans do not. To argue that this is a morally relevant trait that explains the difference between humans and non-humans would need to be demonstrated however, which appears very difficult without simply begging the question. It has also kicked the can down the road as we can ask what trait about this species in particular makes it so special. Some proponents would argue it is not arbitrary as this is the species that we belong to, and the next subsection discusses the flaws in this way of thinking.

\subsection{Speciesism}

Speciesism is the prejudice towards the moral interests of your own species simple because they are your own species. This is very similar to many other prejudices such as racism, sexism, and homophobia. In this subsection we look at how the same reasoning we use to explain why these prejudices are bad can be used to explain why speciesism is bad. In other words, we argue that rejecting speciesism undermines our belief that bigotry is bad.

Why do we choose not to be racist? During the times of slavery, some abolitionist pointed out many black people had achieved great things and showed significant intelligence, and therefore they deserved better treatment (such as not being enslaved). Even if this was not the case, we all accept that slavery would not have been justified. The colour of one's skin is not morally relevant, and those of all races experience pain and suffering similarly. If there is a human trapped in a burning building, we do not need to ask, ``What race are they?", in order to determine whether to save them or not.

Consider instead a general entity trapped in a burning building. What questions about the nature of this entity do we need to ask to determine whether to save it or not? We posit the only question relevant to the moral interests of the entity is whether it can feel pleasure and pain, either physical and mental. This is such a trait where if a human completely lacked it then it would be hard to argue why bludgeoning them with a hammer would be bad for them\footnote{If they did not possess such abilities at the time but may do in the future, such as someone in a coma, you could easily argue you have harmed them by doing this. Arguments about the effects on others could not be used to argue why an action would be bad for them either.}. Any physical sensations experienced would be viewed neutrally. They would have no desires that would be denied. No negative thoughts about the experience at all. It would be as if a rock was bludgeoned.

Determining how and to what extent non-human animals experience pain is a long and scientific endeavour but the consensus is that they do, perhaps with the exception of things like sponges and bivalves. Chickens, pigs, cows, and sheep are closely related to humans enough to have a similar physiology and nervous system, and it is not a great leap to believe they have a similar experience of pain and pleasure. Even mental pains and pleasures that we typically associate with humans are observed in non-human animals, for example a hamster having a taste preference for different foods.

Given this, how can we justify being speciesist, while also believing that we should not be racist, sexist, and homophobic? Our argument for the latter are that race, sex, and sexual orientation are not morally relevant and do not excuse different treatment. It is all based in suffering being bad, and the only way these arguments do not extend apply to non-human animals is if we arbitrarily restrict ourselves to human suffering. Compare this to a racist who restricts themselves to only caring about the suffering of white people as a way of justifying causing harm to non-white people. Try and find an argument against them that would not extend to non-human animals.

If there was a burning building with a human and a chicken trapped inside, I would not expect anyone to run in and save the chicken. Even if there were multiple chickens, there are still strong arguments to prefer to save the human. Accepting speciesism as a form of bigotry does not commit you saving the chickens. For example, you could make arguments based on the greater capacity for suffering of humans, the larger emotional impact their loss would have on others, or the denial of more pleasure they would have had if they continued living to name a few.

It is important to remember during this discussion that we are very rarely comparing human lives to non-human lives, and instead are comparing human taste buds to non-human lives. To conclude that we should not farm animals for food using the practices that we do, we do not need to argue that non-human animals have the same moral worth as humans, we do not even need to argue that they are close. Once we remove the bias of speciesism and weigh up the moral interests impartially, we can make fairer and better justified decisions about how to treat non-human animals.

\subsection{The Veil of Ignorance}

Consider yourself existing among the aether and you are about to enter into the world as some other entity. You could be rich or poor, male or female, human or non-human animal, or even a tree or a chair. This choice will be completely random. What kind of society would you like to be born into?

This is a modified version of John Rawl's veil of ignorance argument which he posed as a way of designing a just and egalitarian society. If a world had a handful of extremely privileged people and everyone else lived in poverty, it would be extremely risky to enter into it randomly as you would likely end up a poor person. It would be preferable to enter into a world where there was a far lower chance of living a life of luxury, but you were very likely to have a modest yet reasonable standard of living. The thought experiment encourages us to remove our biases and think about the bigger picture.

Let us consider our original question. If you were born into the world as a chair, you would be indifferent. Chairs are not sentient, they have no consciousness or thought process, and no moral interests. You would not even be aware of your existence and would therefore necessarily be indifferent to everything, including your destruction. Behind the veil of ignorance about to enter the world you would thus not be worried about being born a chair in a society that kicks, punches, and otherwise abuses chairs. We can apply this logic easily to all non-sentient things that do not experience suffering.

Now suppose you are born into the world as a factory-farmed chicken. It is safe to say that this is not an existence you want. It is a life of abject misery without a single moment of happiness characterised by torture and suffering. Before you write off this world as one you would not want to be born into, consider the fact that you could be born into this world as a human where you would have the opportunity to experience eating chicken. There are around 400 factory-farmed chickens born per person born - do you take the risk? Does the chance of being able to eat chicken not entice you?

Any sane person behind the veil of ignorance would choose to enter a world populated by vegans instead where factory farming did not exist. Not eating animal products suddenly does not seem like such a large price to pay any more. Someone could respond to this by claiming they were not born as a factory-farmed chicken so it does not matter. This would be the same person born into the top 0.0001\% while everyone lived in poverty saying that the poverty did not matter as they were rich. To take this view is to admit you either do not understand the argument or you do not care about the world being fair and egalitarian. If we do not want the world to be a postcode lottery, we should not listen to such people when it comes to designing society.

\subsection{Oppression and Suffering are Bad}

This last argument is a simple test of philosophical consistency and comes in two parts. We have the following simple syllogism.

\begin{itemize}[leftmargin=25mm]
    \item[Premise 1:] Causing easily avoidable suffering is bad
    \item[Premise 2:] It is easy to avoid consuming animal products
    \item[Premise 3:] Consuming animal products causes suffering
    \item[Conclusion:] Consuming animal products is bad
\end{itemize}

The first premise is uncontroversial. To argue otherwise is to argue for pointless sadism for the sake of it