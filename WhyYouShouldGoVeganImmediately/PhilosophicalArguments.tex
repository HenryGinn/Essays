% Introduction. This section aims to do two things: a) demonstrate that animals have moral worth, and b) this moral worth is sufficient to not treat them the way we do.
% Name the trait
    % Explain why you are not racist or homophobic - why can you not apply a similar argument to animals?
% Veil of ignorance
% Do you care about minimising suffering - this entails going vegan
% How much moral worth do animals have
    % What determines moral worth?
    % Where would you place particular animals on a spectrum from a rock to a human? How comfortable would you be hitting either with a hammer?
% Philosophical consistency. Are you against oppression?

% I'll need to define moral worth


\section{Philosophical Arguments}
\label{sec:PhilosophicalArguments}

This section aims to show that the difference in how we treat humans and non-human animals is not justified. We aim to show that the gap is far smaller than currently accepted and that the moral worth of non-human animals is sufficiently high to significantly change how we treat them. This is not a philosophy~101 course and it is assumed that the reader does not need convincing that humans have moral worth and that human suffering is bad as a result. We argue that this should be extended to the claim that suffering is bad without qualification, including the suffering of non-human animals.

We will define the moral worth of an entity as how much weight that entity deserves in decisions about morality. Consider a spectrum with a rock at 0\% and humans at 100\%. We can agree that it would be morally indifferent to hit a rock with a hammer, and morally reprehensible to bludgeon a human with a hammer. We do not believe the rock has any moral interests that need to be considered. Some have argued that animals function as mindless automata like mechanical devices and therefore animals are morally equivalent to rocks. We nip this objection in the bud by dismissing it out of hand because this belief is patently ridiculous.

\subsection{Name the Trait}

The challenge is to name a trait that distinguishes between humans and animals where beings who lack the trait have distinctly less moral worth. This reasoning is then applied to a human who lacks such a trait as a test. We claim that the only morally relevant trait is the ability to experience pain and pleasure, thus creating no categorical ethical distinction between humans and non-human animals. We give some examples of other traits that are typically used to explain the difference in moral treatment of humans and non-human animals and demonstrate why these do not align with our moral intuitions.

As a trivial example to show how the argument goes, suppose we were to claim that the reason human suffering was bad was because we had prehensile digits. We could consider a human with no hands and agree that mercilessly torturing such a person would be a bad thing. From this we conclude that the moral worth of humans is not dependent on having prehensile digits.

Some claim the trait is the ability to reason. However, we can find severely intellectually disabled humans with lower reasoning abilities than pigs, crows, or even smart chickens, yet we are not about to claim that those with profound intellectual disabilities can be treated with wanton abandon. We still believe that bludgeoning such a person to death would be morally reprehensible, therefore we have to go the other way and conclude that a being does not need to be able to reason to have moral worth.

Other proposed traits include understanding one's self as a being that has a past, present, and future, having an internal thought process, and having the ability to communicate with others. The same intellectually disabled person example from before works as a counter to all of these. Another common example is that humans are moral agents, yet there exist people completely uncaring of those around them. Even if all these examples had no human exception there are still many animals who also possess these traits. We challenge the reader to find such a trait or combinations of traits that fully explains the difference in moral treatment between humans and non-human animals that passes this test.

The only trait that one may find is belonging to the species of homo sapiens. It certainly satisfies the conditions that all humans possess it and that all non-humans do not. To argue that this is a morally relevant trait that explains the difference between humans and non-humans needs to be demonstrated, which appears very difficult without simply begging the question. It has also kicked the can down the road as we can ask what trait about this species in particular makes it so special. Some proponents would argue it is not arbitrary as this is the species that we belong to and the next subsection discusses the flaws in this way of thinking.

\subsection{Speciesism}

Speciesism is the prejudice towards the moral interests of a species simply because it is your own species. This is very similar to many other prejudices such as racism, sexism, and homophobia. In this subsection we look at how the same reasoning we use to explain why these other prejudices are bad can be used to explain why speciesism is bad. Conversely, arguments in defense of speciesism can also be applied in defense of other bigotries. In other words, we argue that rejecting speciesism undermines our belief that bigotry is bad.

Why do we choose not to be racist? During the times of slavery, some abolitionist pointed out many black people had achieved great things and showed significant intelligence, and were therefore deserving of better treatment (such as not being enslaved). Even if this was not the case, we all accept that slavery would not have been justified. The colour of one's skin is not morally relevant, and those of all races experience pain and suffering similarly. If there is a human trapped in a burning building, we do not need to ask, ``What race are they?", in order to determine whether to save them or not.

Consider instead a general entity trapped in a burning building. What questions about the nature of this entity do we need to ask to determine whether to save it or not? As alluded to earlier, we posit the only question relevant to the moral interests of the entity is to what extent it can feel pleasure and pain, either physical and mental. This is such a trait where if a human completely lacked it then it would be hard to argue why bludgeoning them with a hammer would be bad for them for example\footnote{If they did not possess such abilities at the time but may do in the future, such as someone in a coma, you could easily argue you have harmed them by doing this. Arguments about the effects on others could not be used to argue why an action would be bad for them either.}. Any physical sensations experienced would be viewed neutrally. They would have no desires that would be denied. No negative thoughts about the experience at all. It would be as if a rock were bludgeoned.

Determining how and to what extent non-human animals experience pain is a long and scientific endeavour but the consensus is that they do, perhaps with the exception of things like sponges and bivalves. Chickens, pigs, cows, and sheep are closely related to humans enough to have a similar physiology and nervous system, and it is not a great leap to believe they have a similar experience of pain and pleasure. Even mental pains and pleasures that we typically associate with humans are observed in non-human animals, for example a hamster having a taste preference for different foods.

Given this, how can we justify being speciesist, while also believing that we should not be racist, sexist, and homophobic? Our argument for the latter are that race, sex, and sexual orientation are not morally relevant and do not excuse different treatment. It is all based in suffering being bad, and the only way these arguments do not apply to non-human animals is if we arbitrarily restrict ourselves to human suffering. Compare this to a racist who restricts themselves to only caring about the suffering of white people as a way of justifying causing harm to non-white people. Try and find an argument against them that would not extend to non-human animals.

If there was a burning building with a human and a chicken trapped inside, no one would expect someone to run in and save the chicken. Even if there were multiple chickens, there are still strong arguments to prefer to save the human. Accepting speciesism as a form of bigotry does not commit you to saving the chickens. For example, you could make arguments based on the greater capacity for suffering of humans, the larger emotional impact their loss would have on others, or the denial of more pleasure they would have had if they were to continue living to name a few.

It is important to remember during this discussion that we are very rarely comparing human lives to non-human lives, and instead we are comparing human taste buds to non-human lives. To conclude that we should not farm animals for food we do not need to argue that non-human animals have the same moral worth as humans, in fact we do not even need to argue that they are close. Once we remove the bias of speciesism and weigh up the moral interests impartially we can make fairer and better justified decisions about how to treat non-human animals.

\subsection{The Veil of Ignorance}

Consider yourself existing among the aether and you are about to enter into the world as some other entity. You could be rich or poor, male or female, human or non-human animal, or even a tree or a chair. This choice will be completely random. What kind of society would you like to be born into?

This is a modified version of John Rawl's veil of ignorance argument which he posed as a way of designing a just and egalitarian society. If a world had a handful of extremely privileged people and everyone else lived in poverty, it would be extremely risky to enter into it randomly as you would likely end up as one of the poor. It would be preferable to enter into a world where there was a far lower chance of living a life of luxury, but you were very likely to have a modest yet reasonable standard of living. The thought experiment encourages us to remove our biases and think about the bigger picture.

Let us consider our original question. If you were born into the world as a chair, you would be indifferent. Chairs are not sentient, they have no consciousness or thought process, and thus no moral interests. You would not even be aware of your existence and would therefore necessarily be indifferent to everything, including your destruction. Subsequently, from behind the veil of ignorance about to enter the world you would not be worried about being born a chair in a society that kicks, punches, and otherwise abuses chairs. We can apply this logic easily to all non-sentient things that do not experience suffering.

Now suppose you are born into the world as a factory-farmed chicken. It is safe to say that this is not an existence you want. It is a life without a single moment of happiness and is characterised by torture and suffering. Before you write off this world as one you would not want to be born into, consider the fact that you could be born into this world as a human where you would have the opportunity to experience eating chicken. There are around 400 factory-farmed chickens born per person born - do you take the risk? Does the chance of being able to eat chicken not entice you?

We posit that given the choice, any sane person behind the veil of ignorance would choose to enter a world populated by vegans where factory farming did not exist instead. Someone could respond to this by claiming they were not born as a factory-farmed chicken so it does not matter. This would be the same person born into the top 0.001\% while everyone else lived in poverty saying that the poverty did not matter as they were born rich. To take this view is to admit you either do not understand the argument or you do not care about the world being fair and egalitarian. If we do not want the world to be a postcode lottery then we should not listen to such people when it comes to designing society.