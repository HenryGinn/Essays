\section{Counter-Arguments}
\label{sec:CounterArguments}

\subsection{Futility}

It is easy to say that going vegan will not make a difference, and this subsection aims to defeat this argument from futility. We will show that it is not futile and that we should be vegan even if it is futile.

Given that there are around 70 billion chickens tortured and killed each year, if we prevent the suffering of one then we have lowered this count by about 0.000000001\%. To use this as an argument that we should not care is to misunderstand percentages. You could use a near identical argument to show that torturing 100 people for 20 years is not that bad as you only affect 0.000001\% of humans. This reasoning is easy to fall into when there is space between you and the animals you are commissioning the torture of.

The individual is worth caring about, even if they are a small part of the total suffering. Suppose someone were torturing you and a passerby said, ``Ehh, me intervening would be futile to the total amount of suffering, therefore I will not bother". You would not accept this argument. We accept that the torture of $x$ individuals is worth stopping. If we then start torturing a further $999x$ individuals then we should still care about reducing the torture of those $x$, even though they only account for 0.1\% of the total. If the number of beings who are suffering gets bigger, why would this change how important it is to reduce the suffering of an individual?

Even if an our efforts are futile, there is still a case for acting out of principle. Suppose you go to a dinner party with a friend. The friend had not done their own cooking however, and instead had kidnapped someone, locked them in the kitchen, and beat them until they complied and prepared the meal. Would you eat this meal? If you do not eat it then all the suffering endured by the enslaved person in the kitchen would be for nothing. Given that the food is already there, you could argue you are not causing additional suffering by eating the meal.

However, the action of eating the meal would show support to the idea that slavery is acceptable as you would implicitly be agreeing with the actions of your friend. Your friend planned to use slavery to do the legwork for the dinner party and you would be going along with that plan. This example demonstrates that we already understand the motivation for doing something out of principle even when it is less than logical. Veganism is similar - it is the refusal to support inflicting extreme suffering.

\subsection{The Cost of Veganism}
\label{sec:TheCostOfVeganism}

We have spent a lot of time looking at the cost of not going vegan, now let us consider in more detail what we lose by going vegan.

The main disadvantage is that you will no longer experience some taste pleasures. This is not to say you will not experience any taste pleasure as there are many meals that are vegan or adaptable into being vegan that taste just as good as a non-vegan meal. You will still be able to enjoy many of the foods you love, just not all of them. Is the experience of eating a beef burger so much better than the experience of eating a plant-based burger that it is worth commissioning a lifetime of immeasurable suffering on a cow for?

The other main cost is convenience as a plant-based diet is less nutritionally convenient than a non-vegan diet. When transitioning you will likely need to plan your meals to ensure you are getting everything you need. Getting all the protein necessary for a healthy diet may be harder than before, and you will likely need to take a B12 supplement. You will need to check labels when buying an item for the first time, and sometimes even search the internet.

The third most common reason I have heard against going vegan is the social cost. At restaurants there may be a very limited choice, for example at a steakhouse, and this can make eating out with friends harder. Some people say they fear social ostracisation and losing friends due to the negative perception of veganism. Personally I would rather not be friends with people who would abandon the friendship because I refuse to fund mass torture and slaughter.

Together these three points cover almost all of the downsides of going vegan. Millions of people willingly go vegan and lead healthy and fulfilling lives. After some time being vegan you will have a new palatte and will not need to plan your nutrition or check labels. You will not miss the foods you used to have, as attested to by the vegan community. After a while it is very easy to be vegan, and the main regret of people who live a vegan lifestyle is that they did not go vegan sooner.

\subsection{High Welfare Farms}

Those who claim they only buy your animal products from high-welfare places are almost certainly wrong. In this subsection we demonstrate why such ``ethical" consumption of animal products is likely not the case.

We see pictures of sheep frolicking in fields on packaging and posters but this is only how the tiniest minority of animals are raised. 99\% of all meat comes from factory farms in the US so less than 1\% of people actually do buy exclusively from high-welfare places on average. Secondly, many products have animal-derived ingredients and it is not easy to determine where they came from or what the welfare standards are. Avoiding these would almost be as restrictive as veganism itself.

Thirdly, high welfare is hard to verify. There are certificates and labels such as Red Tractor and RSPCA approved but these are almost meaningless. Many of these organisations are run by farmers and large animal products monopolies who have a vested interest in ethical-washing their practices and do not hold farms to account. Investigations into almost all farms approved by these groups have been found to be significantly in breach of standards and often even the law.

Many terms such as ``free range" have a far looser legal definition than one would hope for, and other commonly used terms have no regulations over their use. Chickens must not be kept denser than 13~chickens per square meter (about the size of a sheet of A4 paper per chicken), but this number only reduces to 9~chickens per square meter for free range. Neither of these are enforced, and many free range chickens never go outside in practice. All pork can legally be labelled ``free range" in the UK and the EU, independently of what conditions the pigs are kept in. There is a significant effort by animal products manufacturers to keep people in the dark about how animal products are produced and the lawmakers are on their side.

\subsection{Humane Methods of Death and Farming}
\label{sec:HumaneMethodsOfDeathAndFarming}

In discussions about veganism, how animals are killed is often used as an example of a moral wrong. Some argue that everything would be fine if we used more humane methods of slaughter. In this subsection we argue that this is both not true and a distraction. We also counter some objections of the 1\% of people from the previous subsection who actually do buy from ``ethical" farms and claim they do not need to go vegan.

A lot of the narrative and media about animal welfare focuses on shocking treatment, such as male chicks being thrown into a macerator or cows having their throats slit. Death is used as an example of a moral wrong because it is simple, clear cut, and evokes strong emotions, however the chick in the macerator is one of the lucky ones. The time animals spend in the slaughterhouse is brief and almost all of their suffering occurs in factory farms where they are raised. Even if all animals were killed in a relaxing environment via lethal injection, the total suffering over their lifetime would barely be reduced.

On farms where animals are kept in fields there are still many abuses of animals. Examples include tail docking of sheep, forced impregnation, and separating calves from their mothers. Even a farm that used the most ethical practices would be problematic as the animals would still be treated as commodities. What ethical standards would we demand of a human farm before we deemed it morally acceptable? For a less extreme example that avoids objections of sapience, would we accept a dog farm in the western world?

Such hypothetical ultra-humane farms would also be completely impractical. The price would be extortionate and there would be nowhere near enough land or labour for it to be feasible on a large scale. Meat made from insects is also suggested as a low carbon and more ethical method of farming. This would be using animal products out of stubbornness and it is unfathomable that an insect burger would be preferrable over a plant-based burger. Suggesting these as solutions serves only as a distraction and cannot be taken seriously. These farms should only be entertained as an option for the select few with extreme dietary requirements who have no other option in an otherwise vegan world.

\subsection{Other Counter-Arguments}
\label{sec:OtherCounterArguments}

Almost all counter-arguments against can be defeated in a sentence or two. Below we give a whistlestop tour of some other common counters against veganism.

\begin{itemize}
	\item Animal products are needed for health. Meat is not a macronutrient and is not essential for health and studies have found that sufficiently well-planned plant-based diets are healthy at all stages of life, including adolescence and pregnancy. Further, plant-based diets consistently rank among the healthiest diets for humans.
	\item Wild animals eat other animals. Humans in civilised society have easy access to food shops and are not forced to hunt or die. We also have much higher moral agency than a lion for example.
	\item We evolved to eat meat / we have canines. Just because we used to eat meat and have the ability to extract nutrients from animal products does not imply that we have to. This is not relevant to the moral argument given that we are capable of not eating animal products.
	\item Anything about it being bad for the environment. These are almost all bad-faith manipulations of statistics or even complete fabrications.
	\item Veganism does not prevent all suffering against animals. While true, this does not counter the argument that we should minimise the suffering of animals to the highest extent practicable. Animal deaths from farming crops would also be higher to support a non-vegan diet as the animals need to be fed more calories than what they produce. This is also an appeal to futility.
\end{itemize}