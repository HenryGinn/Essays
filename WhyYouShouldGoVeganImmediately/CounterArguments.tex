\section{Counter Arguments}
\label{sec:CounterArguments}

\subsection{Futility}

It is easy to say that going vegan will not make a difference, and this subsection aims to defeat this argument from futility. We will show that it is not futile and that we should be vegan even if it is futile.

Given that there are around 70 billion chickens tortured and killed each year, if we prevent the suffering of one then we have lowered this count by about 0.000000001\%. To use this as an argument that we should not care is to misunderstand percentages. You could use a near identical argument to show that torturing 100 people for 20 years is not that bad as you only affect 0.000001\% of humans. This reasoning is easy to fall into when there is space between you and the animals you are commissioning the torture of. Suppose someone were to torture you and a passerby said, "eh, me intervening would be futile to the total amount of suffering, therefore I won't". You would argue that the harm done to you is considerable and worth stopping. If the number of beings who are suffering gets bigger, why would this change how important it is to reduce the suffering of an individual?

Suppose you go to a dinner party with a friend. The friend had not done their own cooking however, and instead had kidnapped someone, locked them in the kitchen, and beat them until they complied and prepared the meal. Would you eat this meal? If you do not eat it then all the suffering endured by the enslaved person in the kitchen would be for nothing. Given that the food is already there, you could argue you are not causing additional suffering by eating the meal. We are not convinced by these arguments however, and we would not eat this meal out of a matter of principle - we refuse to support slavery. We already understand doing something out of principle even when it is less than logical, and veganism is similar - it is the refusal to support inflicting extreme suffering. The suffering has already happened and so it would be futile to try stop it, yet we should act against it anyway.

\subsection{The Cost of Veganism}
\label{sec:TheCostOfVeganism}

We have spent a lot of time looking at the cost of not going vegan, now let us consider in more detail what we lose by going vegan.

The main disadvantage is that you will no longer experience some taste pleasures. This is not to say you will not experience any taste pleasure, and there are many meals that are vegan or adaptable into being vegan that taste just as good as a non-vegan meal. You will still be able to enjoy many of the foods you love, just not all of them. Is your experience of eating bacon worth commissioning a lifetime of immeasurable suffering on a pig?

The other main cost is convenience. Veganism is less nutritionally convenient than being non-vegan, and when transitioning, you will likely need to plan your meals to ensure you are getting everything you need. Getting all the protein necessary for a healthy may be harder than before, and you will likely need to take a B12 supplement. You will need to check labels when buying an item for the first time, and sometimes even search the internet. In a restaurant you may have very few options to choose from and this can make social occassions harder.

The third most common reason I have heard against going vegan is the social cost. At restaurants there may be very limited choice, for example at a steakhouse, and this can make eating out with friends harder. Some people say they fear social ostracisation and losing friends due to the negative perception of veganism. Personally I would rather not be friends with people who would abandon the friendship because I refuse to fund mass torture and slaughter.

Together these three points covers almost all of the downsides of going vegan. Millions of people willingly do this and lead healthy and fulfilling lives. After some time being vegan you will have a new palette and not need to plan your nutrition or check labels. You will not miss the foods you used to have. After a while it is very easy to be vegan, and the main regret of people who live a vegan lifestyle is that they did not go vegan sooner. The actual reason you are not already vegan is that you were raised as a non-vegan in a predominantly non-vegan society.

\subsection{Humane Methods of Death and Farming}

In discussions about veganism, how animals are killed is often used as an example of a moral wrong. Some argue that everything would be fine if we used more humane methods of slaughter. In this subsection we argue that this is not true, and also that this is a distraction.

Death is used as an example of moral wrong because it is simple, clear cut, and evokes strong emotions. The time animals spend in the slaughterhouse is brief however, and almost all of their life is spent on the factory farm. The farm is where the vast majority of the suffering happens, so even if all animals were killed via lethal injection in a relaxing environment the lifetime suffering is barely reduced.