% You wouldn't support slavery
% You'll be on the right side of history
% If you buy animal products you are complicit
% You would not personally subject the animals to the suffering you indirectly inflict on them
% What standard do you expect from others, and with the information you have, are you holding yourself to that standard?

\section{Emotional Arguments}
\label{sec:EmotionalArguments}

It can be easy to dismiss veganism as not necessary to follow, even if you are philosophically convinced that it is the morally correct action. This section aims to counter that and demonstrate that this is not an issue that can be swept under the rug. These are the reasons why you should care.

\subsection{View on Historical Atrocities}

If you were born 250 years ago, would you have supported slavery? Most of us would like to think that we would not, but in fact you almost certainly would have. The same can be said for other great societal atrocities such as stoning gay people to death and the being complicit with the holocaust. This is because it was the case for most people at the time and you are not some special bastion of moral behaviour who has no blind spots.

If you disagree that you would have been fine with slavery then you are saying you would recognise the moral crime of your time, would deviate from the societal norm, and thus not participate. This is exactly what veganism is now - a recognition that our treatment of animals is a moral crime, and a refusal to participate or support those practices. You may not have been born during a time where you could take a meaningful stance against the slave trade, but you can take a meaningful stance against animal agriculture. Given this, if you are non-vegan the you do not have a leg to stand on to claim that things would have been different during slavery.

This should make you uncomfortable. It is strange to picture us being complicit with such atrocities, but it is exactly the position you are in as a non-vegan. Surprisingly, many people accept that they would have supported slavery and other moral atrocities if they lived in the appropriate era. Are you comfortable with your moral behaviour being dictated by what is accepted by society in the time period you happen to live in. This would mean that the only reason you are not raping or murdering others is because it is looked down upon and you have been conditioned by society not to do these things.

\subsection{The Right Side of History}

Currently we look back on the slave trade in confusion and disgust. It is hard to comprehend how people could have done such things and that they must have been moral monsters. We believe that we are enlightened now and have got things pretty much sorted on the morality front. This is wrong.

The people of the future will look at us in exactly the same way we look at those during the slave trade. They will ask how almost all members of society could support such egrecious moral atrocities. They will want to explain it by thinking that surely the way animals were treated was a secret kept behind closed doors. But it is not a secret, we are aware of what is happening in factory farms and yet actively choose to continue supporting the animal products industry. We are participating and benefiting from one of the largest campaigns of oppression in history.

In the future it is inevitable that our current time will be looked down on as a dark era of humanity. We need to ask whether we want to be on the right side of history or whether we wanted to be lumped in with the people who stoned gay people or kept slaves, and labelled as savages from a more primitive time. The choice of whether we are to be counted among the oppressors or not is ours to make.

\subsection{Oppression}

Do you claim to be against oppression, or do you need to add qualifiers? If someone claimed to be against oppresion simultaneously believed that gingers were an inferior slave race where any subjugation was fair game, we would disagree with their claim. ``Oppression is justifiable as long as those being oppressed possess particular immutable characteristics of my choosing" - this is the view taken by Nazis, assorted hateful lunatics, and also necessarily non-vegans.

Consider the most systemically oppressed human group in western society, for example transgender lesbians of colour who are also poor, disabled, and ugly. How many people think it is acceptable to keep such people in cages or harvest their flesh and secretions? Maybe 5\% as a conservative upper bound? How many think it is acceptable or even actively fund the continuation of the same treatment of non-human animals? It is about 99\%. Non-human animals are so oppressed that the non-vegan does not even recognise them as an oppressed group.

Most recognise that the oppression faced by many humans in our society is completely unacceptable, but their treatment does not even belong on the same scale as the treatment of non-human animals. If someone were to use a microaggression against a human we would call them out on it. If someone were to throw a pig into a cage and lower it into a gas chamber, we would pay them for their trouble (in exchange for some of its flesh of course, what a sane transaction). We are not being consistent with our treatment of oppression. 

Throughout history, oppressed groups have had some opportunity to fight for their liberation despite having very little systemic power. Non-human animals, however, do not have this opportunity. They rely solely on their own oppressors to choose to stop oppressing. They have no voice, no way to organise, and they have almost no freedom, making this the least fair fight for liberation ever. If you are non-vegan then you are one of the oppressors and are complicit in this.

\subsection{Accountability and Distance}

One of the biggest reasons why it is emotionally easy to be non-vegan is because we are separated from the brutal aspects of animal agriculture. This subsection aims to reduce that separation and make clear how accountable a non-vegan is.

Would you stab a pig to death if that was the only way of getting pork? Imagine you are in the meat aisle of a shop and someone ushers you onto a tarpauline, hands you a knife, and presents a tied-up pig. You would feel far more uncomfortable doing this over picking up a packet of meat from the shelves. Even if you arrived at the meat aisle and had to request someone to stab the pig to death on your behalf, you would still feel uncomfortable even though the blood would not literally be on your hands.

The path between you and the factory farm may be indirect, but when you pay for animal products you are commissioning someone to inflict extreme suffering on animals. Imagine giving someone an oral instruction for each stage of the process. For example, ``I want you to stick your arm up the cows arse to hold its cervix in place, and then with the other arm, inject semen into the cervix in order to forcibly impregnate the cow against its will". This is something that will necessarily happen to feed a non-vegan's consumption of cow-derived products\footnote{Almost all factory-farmed cows are impregnated this way.}. Just because you do not need to say the words, it does not mean that it is not happening on your behalf.

Someone may counter that these situations are very different to what actually happens and therefore it is an unfair comparison. Distance between the location of purchase and the location of slaughter is morally irrelevant, but how does the shared accountability change things? The situation is a lot closer to 20 people collectively requesting the pig to be slaughtered. This argument does not work because if a person eats one pigs worth of pork then on average they would have commissioned a pig to death.

The police officer loading people onto trains in the Holocaust could also claim they played a small part and that the outcome would have been the same if they had refused. We do not take this defense of shared accountability seriously as this only implies that we should not blame the entirety of the Holocaust on them but they still share some blame for participating. We also note that if almost all the police officers had refused to participate then this would have significantly hindered logistics. Collective action is powerful and collectives are made of individuals.