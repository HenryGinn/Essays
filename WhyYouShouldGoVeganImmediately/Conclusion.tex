% How much do you care about being philosophically consistent?
% Other reasons why you should be vegan unrelated to ethical veganism

\section{Conclusion}

Not covered here is the evidence of the suffering and all the other reasons for veganism outside of ethical veganism. Here we signpost and summarise those points.

The suffering cannot be understood without seeing it, and if you do not feel comfortable watching a documentary on factory farming then the point has already been made. Keep in mind while watching that the footage is not cherry-picked and that almost all people willingly fund such suffering and demand it continues. It would be hard to find any justification for what is inflicted on animals in factory farms, yet the non-vegan must argue that taste pleasure and nutritional convenience alone are sufficient.

As an example of the conditions in factory farms, farmers can only spend brief moments inside the large shed where animals are kept. There is so much ammonia in the air that eye and respiratory problems are still a high risk for them even with these precautions. The animals get no respite from this environment. Below is a list of documentaries and we ask the reader to watch one of them.

\begin{itemize}
	\item Dominion: \url{}
	\item Land of Hope and Glory: \url{}
	\item 
\end{itemize}

Each of the following reasons would be enough on their own to justify going vegan.

\begin{itemize}
	\item Climate change: The animal products industry is one of the biggest causes of climate change, and going vegan is one of the highest impact changes you can make to lower your carbon footprint.
	\item Pollution: The animal products industry is also responsible for a significant portion of air and water pollution.
	\item Ocean plastic: More than half of the Great Pacific Garbage Patch (three times the size of France) is from fishing nets. The fishing industry is responsible for a large portion of ocean plastic.
	\item Deforestation: Clearing room for animals or crops to feed the animals is the leading cause of deforestation due to poor land per calorie efficiency.
	\item Biodiversity: The animal agriculture industry is the leading cause of species loss.
	\item Unsustainability: The earth would need to be three times larger if everyone were to eat the diet of the average American. We already have enough land allocated for food to feed everyone on Earth if they were vegan.
	\item Zoonotic diseases: Thousands of animals in one place in terrible conditions is a perfect storm for creating zoonotic diseases.
	\item Antibiotic resistance: a significant proportion of all antibiotics are used by the animal agriculture industry.
\end{itemize}

The actual reason non-vegans are not already vegan is that they were raised as a non-vegan in a predominantly non-vegan society. Similarly, most Christians are Christian because they were born into a Christian family in a predominantly Christian society. They did not reason their way into religion in the same way that non-vegans did not reason their way into non-veganism. As atheists we can recognise the absurdity in Christianity, but a non-vegan born in a non-vegan world does not see the absurdity in non-veganism. In this essay we ask the reader to question the default view of society and judge whether it stands up.

You will not beat the argument for veganism on the facts and logic. There are simply too many strong reasons for it, and the negative impact on your life is not enough to counteract them outside of people with extremely restrictive dietary requirements. Upon reading this and watching a documentary of life in factory farms you will no longer be able to claim ignorance. There will be several choices:

\begin{itemize}
	\item The Unlucky: Be born with dozens of allergies and intolerances (this probably is not you).
	\item The Gymnast: Somehow argue against all the arguments for veganism (good luck).
	\item The Moron: Claim you are correct without countering any arguments. You will not be able to claim to follow the science or rational argument, and you will be no better than the MAGAs.
	\item The Scumbag: Accept that you should go vegan but continue funding torture and slaughter.
	\item The Complicit: Accept that it is all very bad, but that you going vegan will not change anything.
	\item The Enlighted: Go vegan.
\end{itemize}